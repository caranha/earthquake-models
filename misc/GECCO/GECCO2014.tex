% This .tex source is an example which *does* use
% the .bib file (from which the .bbl file % is produced).
% REMEMBER HOWEVER: After having produced the .bbl file,
% and prior to final submission, you *NEED* to 'insert'
% your .bbl file into your source .tex file so as to provide
% ONE 'self-contained' source file.
%

\documentclass{sig-alternate}

\usepackage{algorithm}
\usepackage{algorithmic}

\begin{document}

\conferenceinfo{GECCO'14}{July 12-16,2014, Vancouver, BC, Canada.}
\CopyrightYear{2014}
\crdata{TBA}
\clubpenalty=10000
\widowpenalty=10000

\title{Is it possible to generate good Earthquake Risk Models using Genetic Algorithms?}

\numberofauthors{4} 

\author{
\alignauthor Blind Author 1\\
       \affaddr{University 1}\\
       \affaddr{Address 1}\\
       \affaddr{Address 2}\\
       \email{e-mail@address.com}
\alignauthor Blind Author 2\\
       \affaddr{University 1}\\
       \affaddr{Address 1}\\
       \affaddr{Address 2}\\
       \email{e-mail@address.com}
\and
\alignauthor Blind Author 3\\
       \affaddr{University 1}\\
       \affaddr{Address 1}\\
       \affaddr{Address 2}\\
       \email{e-mail@address.com}
\alignauthor Blind Author 4\\
       \affaddr{University 1}\\
       \affaddr{Address 1}\\
       \affaddr{Address 2}\\
       \email{e-mail@address.com}
}

\date{29 January 2014}


\maketitle
\begin{abstract}
Understanding the mechanisms and patterns of earthquake occurrence is of
crucial importance for assessing and mitigating the seismic risk. In this
work we analyze the viability of using Evolutionary Computation (EC) as a
means of generating models for the occurrence of earthquakes. Our proposal
is made in the context of the "Collaboratory for the Study of Earthquake
Predictability" (CSEP), an international effort to standardize the study
and testing of earthquake forecasting models.

We use a standard Genetic Algorithm (GA) with real valued genome, where
each allele corresponds to a bin in the forecast model. The design of an
appropriate fitness function is the main challenge for this task, and we
test three different proposals, all based on the log-likelihood of the
candidate model against the training data set.

The resulting forecasting models are compared with statistical models
traditionally employed by the CSEP community, using data from the
Japan Meteorological Agency (JMA) earthquake catalog. Our results
indicate promise for the use of GA as basis for constructing
statistical earthquake forecast models. Based on these results, we
identify research directions that we consider deserve more attention
from the EC community.
\end{abstract}

% A category with the (minimum) three required fields
\category{I.2.8}{Artificial Intelligence}{Problem Solving, Control
  Methods and Search Heuristic Methods}
%A category including the fourth, optional field follows...
\category{J.2}{Computer Application}{Physical Sciences and
  Engineering}[Earth and Atmospheric Sciences]

\terms{Real World Applications}

\keywords{Earthquakes, Risk Models, Genetic Algorithms, Forecasting}

\section{Introduction}

(Goals of the introduction, 1: Establish significance of Earthquake
studies, and define the ``Forecast Model'' problem. Define the CESP
and RELM initiatives. 2: Explain the difficulty of the Forecast Model
problem and explain the motivation for use of Genetic Algorithms. 3:
This is a little explored problem in GA, establish the scope of this
paper. 4: Summarize the contribution and results of this paper)

** Studying earthquakes is important to society. Earthquakes are
natural events that can cause large amounts of damage both in terms of
loss of human life, and material damages. By understanding and
measuring the risk of earthquakes, it would be possible to guide
policy decisions regarding construction regulations, and the
deployment of disaster relief infrastructure.

** However, the specifics of earthquake generation mechanisms are not
yet completely understood. In this sense, the development of accurate
models for the risk assessment of earthquake activity is a very active
area of study.

** In the study of Earthquake predictability, we highlight the RELM
(Relative Earthquake Log-likelihood Model) and the CSEP

**** Evolutionary algorithms have shown themselves to be very
effective in hard problems, such as (...). However, the use of EC
techniques in seismology studies, and in particular in the development
of earthquake risk models, is extremely scarse.

** We believe that a real contribution can be make by the use of
evolutionary techniques in this field. Therefore in this paper we ****
make a tutorial on earthquakes,

*** We illustrate these ideas by designing a simple genetic algorithm
that creates an earthquake prediction model as defined by the CSEP. We
compare this algorithm with a ``skilless'' model (random model), and
the RI model. Using Data from the JMA catalog between 2002 and 2010

** Outline of the paper
In section 2, we describe the Earthquake predictability problem.

\section{The Earthquake Predictability Problem}

Earthquake Risk Analysis is a huge area, which includes the
development of models for earthquakes, and many other fun things. We
are concentrating on the development of Earthquake Predictability
Model. 

In this problem, we want to create a model that, for a certain area
during a certain time in the future, defines expected number of
earthquakes and their respective power.

   - This area is huge, we are focusing in a small sub area, which is
   nonetheless very intersting.

There are two main types of models: In the statistical models, the
forecast is based mainly on the distribution of earthquakes of
different magnitudes in the past. In Geophisical models, a theoretical
model of how the seismic energy is released, and forecasts are made
based on these models.

%%% IMAGE HERE: A forecast model with the earthquakes in it.

\subsection{RELM And CSEP}

Description of the CSEP - collaboratory for the study of earthquake
predictability.  It stabilish ways to analyse and compare earthquake
predictability experiments.

The CSEP define a series of standardized tests, such as RELM based
tests (R Test, N Test, ETC)~\cite{Schorlemmer2007}, and also alarm
based tests, such as the Molchamm Diagram, the Area Skill
Score~\cite{Zechar2010}, etc.

The CSEP determines ``testing windows'' -- one day, one month, three
months, one year, three years.

\subsection{Earthquake Predictability Model Testing}

Model Definition -- bins with a number of earthquakes in them.

Log Likelyhood comparison - Explanation

CSEP tests - Explanation

\subsection{The RI algorithm}

Outline of the RI Algorithm~\cite{Nanjo2011}, and how it is often used
as a benchmark


\section{Evolutionary Computation for Earthquake Risk Analysis}

%% TODO: Read Yuri's Paper list and incorporate into this text.
Compared to other application fields, there are relatively few
applications of Evolutionary Computation to seismology problems.
We identify 

In the field of seismology and earthquake risk analysis, the few cases
of Evolutionary Algorithm approaches have usually taken one of two
forms. EC is often used to estimate parameter values for seismological
models.  These models can be used to describe and understand
earthquakes.

For example, Ramos~\cite{Ramos2011} uses Genetic Algorithms to decide
the location of sensing stations in a seismically active area in
Mexico. Nicknam et. al~\cite{Nicknam2010} and Kennett and
Sambridge~\cite{Kennett1992} used evolutionary computation to
determine the Fault Model parameters (such as epicenter location,
strike, dip, etc) of a given earthquake. Evolutionary Computation has
also been used to estimate the peak ground accelleration of
seismically active areas~\cite{Kermani2009, Cabalar2009}.

EC has also been used very few times as a method to generate seismic
forecast model. One such approach is described by Zhang and
Wang~\cite{Zhang2012}, where they fine tune a Neural Network with a
GA, and then use this system to produce a forecast. Unfortunately that
work did not provide enough information to reproduce the proposed
GA+ANN system or their results.

\section{A Forecast Model Using Genetic Algorithms} %% DONE

To investigate the ability of Evolutionary Computation to generate
earthquake forecast models, we design and test a simple Genetic
Algorithm. Let us call this system the \emph{GAModel}.

An individual in GAModel will encode a forecast model as defined in
the CSEP framework~(\ref{CSEP_definition}). The population will be
trained on earthquake occurrence data for a fixed training period. The
best generated individual will be compared with a model generated by
the RI algorithm, and a skilless (random) model. This comparison is
based on earthquake occurrence data for a test period immediately
posterior to the training period.

By encoding the entire forecast model as one individual, we identified
two main concerns while designing GAModel: First, as the forecast
model contains a large number of bins, the genome of an individual
will be respectively large. Evolutionary operators and parameters must
be chosen carefully to guarantee convergence in a reasonable time, and
avoid local optima.

Secondly, the design of the fitness function deserves a lot of
attention, to avoid the risk of the system overfitting to the training
data.

\subsection{Genome Representation} %% DONE

%% TODO: Define bin here (as well as in the problem definition), if
%% there is enough space

In GAModel, each individual represents an entire forecast model. The
genome is a real valued array $X$, where each element corresponds to
one bin in the desired model (the number of bins $n$ is defined by the
problem). Each element $x_i \in X$ can take a value from $[0,1)$. In
  the initial population, these values are sampled from a uniform
  distribution.

In the CSEP framework, a model is defined as a set of integer valued
expectations, corresponding to the number of predicted earthquakes for
each bin. To convert from the real valued chromossome to a integer
forecast, we use a modification of the Poisson deviates extraction
algorithm from~\cite{NumericalRecipes}~(Chapter 7.3.12).

\begin{algorithm}
  \caption{Obtain a poisson deviate from a $[0,1)$ value}
  \label{InversePoisson}
  \begin{algorithmic}
    \STATE Parameters $0 \leq x < 1, \mu \geq 0$
    \STATE $L \gets \exp{(-\mu)}, k \gets 0, prob \gets 1$
    \REPEAT 
    \STATE $\text{increment }k$
    \STATE $prob \gets prob*x$
    \UNTIL{$prob > L$}
    \RETURN $k$
  \end{algorithmic}
\end{algorithm}

In Algorithm~\ref{InversePoisson}, $x$ is the value taken from the
chromossome, and $\mu$ is the average number of earthquakes observed
across the entire training data. Note that in the original algorithm,
$k-1$ is returned. Because the log likelihood calculation used for
model comparison discards forecasts that estimate $0$ events in bins
where earthquakes are observed, we modify the original algorithm here
to make sure all bins estimate at least one event.

%% TODO: If We have space -- Explain that \mu is resized if the time
%% intervals are different between the training and the testing data
%% sets.

\subsection{Fitness Function}

The main challenge when applying an Evolutionary Computation method to
any new application domain is usually the definition of an appropriate
fitness function. Accordingly, the largest part of our effort in this
work was to define a good fitness function for GAModel.

\subsubsection{Simple Log Likelihood Fitness Function} %% DONE

Our first candidate was the log-likelihood between the forecast
generated by an individual and the observed earthquakes in the
training data, as described by Schorlemmer
et. al.~\cite{Schorlemmer2007}. In simple terms, the log likelihood is
a measure of how close a forecast is to a given data set.

Let $\Lambda = \{\lambda_1, \lambda_2, \dots, \lambda_n | \lambda_i
\in \mathbb{N}\}$ be a forecast with $n$ bins. In this definition,
$\lambda_i$ is the number of earthquakes that is forecast to happen in
bin $i$. To derive $\Lambda_X$ from an individual $X = \{x_1, x_2,
\dots, x_n | 0 \leq x_i < 1\}$, we calculate each $\lambda_i$ from
$x_i$ using Algorithm \ref{InversePoisson}.

Let $\Omega = \{\omega_1, \omega_2, \dots, \omega_n | \omega_i \in
\mathbb{N}\}$ be the observed earthquakes in the training data. The
log likelihood between an individual's forecast $\Lambda_X$ and the
observed data $\Omega$ can be calculated as:
\begin{equation}
  L(\Lambda_X|\Omega) = \sum_{i=0}^n {-\lambda_i +
    \omega_i*\ln(\lambda_i)-ln(\omega_i!)}
\end{equation}
There are two special cases that arise when any $\lambda_i = 0$. If
$\lambda_i = 0$ and $\omega_i = 0$, then the value of the sum for that
element is $1$. If $\omega_i > 0$, then $L(\Lambda_X|\Omega) =
-\infty$ and the forecast must be discarded. For more details on
this, see~\cite{Schorlemmer2007}.

In the Simple Log Likelihood fitness function, the value of
$L(\Lambda_X|\Omega)$ is taken as the fitness value of the individual.

Early testing with the Simple Log Likelihood function showed that
GAModel had a very strong tendency to overfit to the training
data. This is natural, since there are differences between the
seismocity of a larger period and a shorter one. 

To avoid this overfitting, we have developed two alternative fitness
functions.

\subsubsection{Simulated Log Likelihood Fitness Function} %% DONE

In the Simulated Log Likelihood fitness function, we generate $k$
``simulations'' from the observed data $\Omega$, and compare an
individual's forecast with all the simulated data.

Let $\hat\Omega =
\{\hat\omega_1,\hat\omega_2,\dots,\hat\omega_3
|\hat\omega_i\in\mathbb{N}\}$ be a simulated data set generated from
$\Omega$. Each $\hat\omega_i$ is taken randomly from a Poissonian
distribution with mean $\omega_i$.

To calculate the Simulate Log Likelihood fitness, we calculate the Log
Likelihood of the forecast $\Lambda_X$ generated by an individual $X$
against each of the $k$ simulated data sets
($L(\Lambda_X|\hat\Omega_1), \dots, L(\Lambda_X|\hat\Omega_k)$). We
use the lowest of the $k$ log likelihood values as the fitness of $X$.

The idea behind this fitness function is that by training against a
number of similar, but slightly different training data sets at the
same time, we might be able to avoid overfitting the forecast to the
observed training data set.

\subsubsection{Time-slice Log Likelihood Fitness Function} %% DONE

In the time-slice log likelihood fitness function we break up the
training data set into smaller \emph{slices}. These slices are based
on the chronology of the earthquakes contained in the training
catalog. The duration of each slice is the same as the duration of the
test catalog.

Let's consider an example where the target period for the forecast is
one year, from 1/1/2014 to 1/1/2015, and the training data is taken
from the 10 year period of 1/1/2004 to 1/1/2014. The time-slice log
likelihood fitness function will divide the training data into ten
1-year periods, from 2004 to 2005, 2005 to 2006, and so on. 

When an individual $X$ is evaluated, we calculate the log likelihood
of its forecast $\Lambda_X$ against each of the time slices
($\Omega_{2004}, \Omega_{2005}, \dots, \Omega_{2013}$). The lowest
value is used as the fitness of $X$.

The idea behind this fitness function is that the catalog data
available for training will normally span a period of time much longer
than the desired forecast. By breaking the training data into smaller
periods, we are trying to make the evolutionary algorithm learn any
time-repeating pattern that might exist in the data.

\subsection{Evolutionary Operators and Parameters} %% DONE

GAModel uses a regular generational genetic algorithm. For selection,
we use Elitism and Tournament selection. 

For the crossover operator we use \emph{aNDX}, recently proposed by
Someya~\cite{Someya2013}. If the value of a chromossome after the
crossover falls outside the $(0,1]$ boundary, it is truncated to these
  limits. For the mutation operator, we sample entirely new values
  from $(0,1]$ for each mutated chromossome.

\begin{table}[!h]
  \begin{center}
  \begin{tabular}{|l|r|}
    \hline
    Population Size & 1000\\
    Generation Number & 500\\
    Elite Size & 1\\
    Tournament Size & 50\\
    Crossover Chance & 0.9\\
    aNDX parents & 4\\
    aNDX zeta & 0.5\\
    aNDX sigma & 1\\
    Mutation Chance (individual) & 0.8\\
    Mutation Chance (chromossome) & (genome size)$^{-1}$\\
    \hline    
  \end{tabular}
  \end{center}
  \caption{Parameters for GAModel}
  \label{GAParameters}
\end{table}

The parameters used for the evolutionary computation are described in
Table~\ref{GAParameters}. At this stage, we are not yet particularly
worried with convergence speed of the system. Because of this, not a
lot of effort was spent fine tuning these paremeter's values. Instead,
these values were choosen by trial and error on a data region not used
in the experiments of section~\ref{data}, until an acceptable
convergence time was found.

%% TODO: Overfitting comparison between the three fitness functions.

\section{Experiments}
%% TODO: Introduction to Experiments section
%% TODO: Goals of our experiments
%% Outline of the experiments

\subsection{Experimental Data}\label{data}

The data used in these experiments comes from the Japan Meteorological
Agency's (JMA) catalog. The catalog lists earthquakes recorded by the
sensing station network in Japan, along with the time, magnitude,
latitude, longitude and depth of the hypocenter.

The part of the catalog used in this work spans a period from 1997 to
2013, and has over 300.000 recorded earthquakes over the Japanese
archipelago. In order to better understand the predictive power of
GAModel, we divide the catalog into 5 areas, which are used in the
experiments below. The relative position and size of these areas can
be seen in Figure~\ref{areamap}.

\begin{enumerate}
  \item {\bf All Japan:}
  \item {\bf Sendai:}
  \item {\bf Kantou:}
  \item {\bf Kansai:}
  \item {\bf Touhoku:}
\end{enumerate}

%% TODO: Geographical Description of the Data (depth, number of bins, reasoning)
%% TODO: Figures of the locations we chose for the experiments

\subsection{Experimental Design} %% DONE

To test the ability of a Genetic Algorithm to generate an effective
earthquake risk model, we perform a forecast experiment. In this
experiment we compare three forecasts: a randomly generated forecast,
one generated by the RI algorithm, and a group of forecasts generated
by the proposed method.

We define a ``scenario'' as a region and a forecast period. In this
experiment, we will test 5 regions (all japan, pacific, kanto, tohoku,
kansai), and 8 one-year periods (from 2004 to 2012), for a total of 40
scenarios.

Each scenario has a corresponding training window, which consists of
the 6 years prior to the scenario's forecast time. This training
period is used by the RI algorithm and by the proposed method to
generate their respectives forecasts. For the sake of statistical
testing, we generate 20 forecasts using the GAModel. Unless noted
otherwise, all results reported are an average of these runs.

%% TODO: Remove ASS if we end up not using it.
We compare the forecasts in three different ways: Using the
log-likelihood between the forecast and the testing data, Using the
ASS value, and visually by comparing the forecast maps of some
significative scenarios.

\subsection{Forecast Model comparison}

%% TODO: Table with all the log likelihood results
%% Indicate whether the difference between RI and GA is statistically significant

%% TODO: Table with the ASS results
%% Indicate whether the difference between RI and GA is statistically significant

%% TODO: Choose some scenarios (maybe 3?) to visually compare

%% TODO: If there is not enough space, indicate that extra results
%% will be left as additional material

\section{Conclusion}

%% TODO: Conclusions and Future Work

-- We choose the parameters hapzardly - a more careful parameter
selection might improve the result (or at least the running time)

-- Conclusions and Future Work
   - Development of the current idea
   - Involvement of the community (De-clustering, Parameter optimization)
   - Please let's work with earthquakes!
   - GP
\subsection{Future Work}

\section*{Acknowledgements}

TODO: Check with Bogdan if we need to provide Acknowledgements for the
JMA data.

% initial runs of your .tex file to
% produce the bibliography for the citations in your paper.
\bibliographystyle{abbrv}
\bibliography{earthquake}

%\balancecolumns
\end{document}
